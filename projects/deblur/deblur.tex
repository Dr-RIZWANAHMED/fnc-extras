\documentclass[11pt,letterpaper]{article}
\usepackage[T1]{fontenc}
\usepackage{amsmath}
\usepackage{amsfonts}
\usepackage{amssymb}
\usepackage{graphicx}

\usepackage[charter]{mathdesign}
\usepackage{fullpage}
\pagestyle{empty}

\usepackage{amsmath}
\usepackage{bm}

\newcommand{\nat}{\mathbb{N}}          % Natural numbers
\newcommand{\integer}{\mathbb{Z}}      % Integers
\newcommand{\real}{ {\mathbb{R}} }     % Reals
\newcommand{\float}{ {\mathbb{F}} }     % Reals
\newcommand{\rmn}[2]{ \mathbb{R}^{#1\times#2} }     % Reals
\newcommand{\complex}{ {\mathbb{C}} }  % Complex
\newcommand{\macheps}{\ensuremath \varepsilon_{\text{mach}}}

\renewcommand{\Re}{\operatorname{Re}}
\renewcommand{\Im}{\operatorname{Im}}

% Boldface vectors
\newcommand{\bff}{\bm{f}}
\newcommand{\bfF}{\bm{F}}
\newcommand{\bfw}{\bm{w}}
\newcommand{\bfv}{\bm{v}}
\newcommand{\bfe}{\bm{e}}
\newcommand{\bfc}{\bm{c}}
\newcommand{\bfp}{\bm{p}}
\newcommand{\bfq}{\bm{q}}
\newcommand{\bfr}{\bm{r}}
\newcommand{\bfs}{\bm{s}}
\newcommand{\bfu}{\bm{u}}
\newcommand{\bfb}{\bm{b}}
\newcommand{\bfx}{\bm{x}}
\newcommand{\bfy}{\bm{y}}
\newcommand{\bfg}{\bm{g}}
\newcommand{\bfh}{\bm{h}}
\newcommand{\bfz}{\bm{z}}
\newcommand{\bfa}{\bm{a}}
\newcommand{\bft}{\bm{t}}
\newcommand{\bfd}{\bm{d}}
\newcommand{\bfalpha}{\bm{\alpha}}
\newcommand{\bfeps}{\bm{\varepsilon}}
\newcommand{\bfdelta}{\bm{\delta}}
\newcommand{\bfzero}{\bm{0}}
\newcommand{\eye}[1]{\bfe_{#1}}

% Boldface matrix
\newcommand{\m}[1]{\bm{#1}}
\newcommand{\mA}{\m{A}}
\newcommand{\mL}{\m{L}}
\newcommand{\mF}{\m{F}}
\newcommand{\mU}{\m{U}}
\newcommand{\mJ}{\m{J}}
\newcommand{\mP}{\m{P}}
\newcommand{\mQ}{\m{Q}}
\newcommand{\mR}{\m{R}}
\newcommand{\mD}{\m{D}}
\newcommand{\mS}{\m{S}}
\newcommand{\mB}{\m{B}}
\newcommand{\mC}{\m{C}}
\newcommand{\mE}{\m{E}}
\newcommand{\mG}{\m{G}}
\newcommand{\mH}{\m{H}}
\newcommand{\mV}{\m{V}}
\newcommand{\mW}{\m{W}}
\newcommand{\mX}{\m{X}}
\newcommand{\mZ}{\m{Z}}
\newcommand{\mK}{\m{K}}
\newcommand{\mM}{\m{M}}

\newcommand{\meye}{\m{I}}

\newcommand{\ee}[1]{\times 10^{#1}}
\newcommand{\jac}[2]{\frac{\bfd \bm{#1}}{\bfd \bm{#2}}}
\newcommand{\diag}{\operatorname{diag}}
\newcommand{\fl}{\operatorname{fl}}
\newcommand{\circop}[1]{\makebox[0pt][l]{$\bigcirc$}\hspace{1pt}#1}
\newcommand{\myvec}{\operatorname{vec}}
\newcommand{\unvec}{\operatorname{unvec}}
\newcommand{\kron}[2]{#1 \otimes #2}


\begin{document}
	
\begin{center}
  \bf 
  Project: Deblur, denoise, delight!
\end{center}
	
Recall that an image may be represented as an $m\times n$ matrix $\mX$ of pixel intensities. A fair way to simulate blurring the image is by
\begin{equation}
\label{blurmat}
\mZ =  (\mB_m)^p \mX \, (\mB_n^T)^p,
\end{equation}
where $p$ is a positive integer and each $\m{B}_k$ is a $k\times k$ symmetric tridiagonal matrix with $1/2$ all along the main diagonal, and $1/4$ all along both the sub- and superdiagonals. As can easily be verified, the mapping from $\mX$ to $\mZ$ is linear, so if we represent the images as vectors (say, by stacking columns), then $\bfz=\mG \bfx$. The matrix $\m{G}$ must be $mn\times mn$, and can be found explicitly, but we consider it to be unavailable. 

Suppose that the image $\mZ$ is derived from both blurring of $\mX$ and some added noise. We can write this as 
\begin{equation}
\label{forward}
\bfz = \mG \bfx + \bfy, \quad \text{for some $\bfy$ with $\|\bfy\|^2=\delta^2$},
\end{equation}
where $\delta$ is a small(?) number. Our goal is the reconstruction problem: for known $\bfz$, $\mG$, and $\delta$, return the best (most likely) $\bfx$. An optimization is essential, since there are many possible solutions of~\eqref{forward}. 

Truly random noise will look like static, with lots of value jumps from pixel to pixel. Real images, on the other hand, usually have large patches of slowly-varying values. Consider the $k\times k$ matrix 
\begin{equation}
\mR_k = 
\begin{bmatrix}
  -2 & 1      &        &        &    \\
  1  & -2     & 1      &        &    \\
     & \ddots & \ddots & \ddots &    \\
     &        & 1      & -2     & 1  \\
     &        &        & 1      & -2 \\
\end{bmatrix}.
\end{equation}
$\mR_k$ is a scaled finite-difference approximation of the second derivative, so $\|\mR_k\bfu\|$ is a measure of the ``roughness'' of the vector $\bfu$. The 2D equivalent of this process on an image matrix $\mX$ is to apply roughening from both the left and the right, in the form 
\begin{equation}
  \label{roughmat}
   \mR_m\mX + \mX\mR_n^T.
\end{equation}
This too is a linear transformation of the image $\mX$, which we denote by $\m{L}\bfx$ for an unspecified matrix $\mL$. Minimizing roughness, as measured by $\|\mL\bfx\|$, seems like a reasonable optimization criterion. 

We now have a complete problem to state:
\begin{equation}
   \label{consopt}
   \min \| \mL \bfx\|^2 \; \text{over all $\bfx$ satisfying $\|\mG\bfx-\bfz\| = \delta$.}
\end{equation}
This is a multidimensional minimization subject to a scalar constraint function, a problem nicely changed to an unconstrained form by the method of Lagrange multipliers. Introducing the unknown scalar $\lambda$, we obtain the simultaneous equations
\begin{align}
    (\mL^T\mL + \lambda \mG^T\mG)\bfx &= \lambda \mG\bfz \label{lineq}\\
    \|\mG\bfx - \bfz\| - \delta & = 0. \label{nonlineq}
\end{align}

Equation~\eqref{lineq} is an $mn\times mn$ linear system that defines $\bfx$ when $\lambda$ is given. We could pose~\eqref{nonlineq} as a rootfinding problem for $\lambda$. However, we rarely know a good value for $\delta$ in advance, so it makes more sense to regard the solution $\widehat{\bfx}$ of~\eqref{lineq} as a function of a tunable parameter $\lambda$. As $\lambda\to0$, greater emphasis is placed on minimizing roughness (increased smoothing), and as $\lambda\to\infty$, more emphasis is on deblurring (increased sharpness).

The matrices $\mL$ and $\mG$ are never needed for these computations. It's possible to show that $\mL$ and $\mG$ are both symmetric, so $\mL^T\mL\bfx=\mL(\mL\bfx)$ and $\mG^T\mG\bfx=\mG(\mG\bfx)$. For implementation, the multiplication $\mL\bfx$ or $\mG\bfx$ is achieved by reshaping $\bfx$ to the image matrix $\mX$,  applying~\eqref{roughmat} or~\eqref{blurmat}, respectively, and then reshaping back into a vector. Hence, given $\lambda$, the operator on the left side of~\eqref{lineq} can be implemented by writing a function that can be passed to \texttt{gmres} in lieu of a matrix.

A final note: Adding a constant to all the entries of $\mZ$ can improve the reconstruction. A typical choice is to make the mean value of the pixels equal to zero. This doesn't affect the plot using \texttt{imagesc}, which maps min and max pixel values to black and white. 

\subsection*{Project assignment}
\label{sec:project-assignment}

Submit a zip archive with files as directed here. Don't forget to convert an imported image to a 2D matrix of class \texttt{double}. \textbf{You should use \texttt{imagesc} for plotting all images, and after each image plot use}
\begin{verbatim}
    colormap(gray(256))), axis equal
\end{verbatim}

\begin{description}
\item[\textbf{Objective 1.}] Derive~\eqref{lineq}--\eqref{nonlineq} from~\eqref{consopt}. You will need to consult a calculus book, and to take the gradients of expressions like $\mC^T \mC \bfu$ and $\bfu^T \bfc$ with respect to $\bfu$. 

For this objective, submit the following:
\begin{itemize}
\item A PDF, \texttt{objective1.pdf}, typeset nicely or scanned from a neatly written document and showing the derivation in convincing detail.
\end{itemize}

\item[\textbf{Objective 2.}]  Write two functions:
\begin{verbatim}
    function Z = blur(X,p)
    function Z = roughen(X)
\end{verbatim}
They implement the equivalent of $\mG\bfx$ and $\mL\bfx$, respectively, but by operating on $m\times n$ images via~\eqref{blurmat} and~\eqref{roughmat}. In practice, both $m$ and $n$ are small enough that it's not important to use sparse matrices. 
You can test your functions on the image \texttt{checkeredflag} available on the website. 

For this objective, submit the following:
\begin{itemize}
\item Files \texttt{blur.m} and \texttt{rough.m}. Each should be self-contained. 
\item Two PNG images. One shows the original and blurred (with $p=2$) checkered flag images side by side, and the other shows the original and roughened images side by side. 
\end{itemize}

\item[\textbf{Objective 3.}] Write a function
\begin{verbatim}
    function y = linop(x,lambda,p,m,n)
\end{verbatim}
The inputs are a vectorized $mn\times 1$ image $\bfx$, a positive scalar value for $\lambda$, and the size $m\times n$ of the original image. The output is $\bfy=(\mL^T\mL + \lambda \mG^T\mG)\bfx$, computed using calls to \texttt{blur} and \texttt{roughen}.

For this objective, submit the following:
\begin{itemize}
\item The file \texttt{linop.m}. It should depend only on native MATLAB commands and \texttt{blur} and \texttt{roughen}.
\item A PNG file showing the result of applying \texttt{linop} to the checkered flag image with $\lambda=8$, $p=4$.
\end{itemize}

\item[{Objective 4.}] Download \texttt{fonzie.png} from the website and import it as the degraded image $\mZ$. For $p=3$ and $\lambda=0.05,0.2,1,5$, apply \texttt{gmres} using \texttt{linop} to reconstruct the image $\widehat{\mX}$ (the matrix form of the solution of~\eqref{lineq}).

For this objective, submit the following:
\begin{itemize}
\item Four PNG images, each showing one of your reconstructions with the value of $\lambda$ in the title.
\end{itemize}


\item[Objective 5.] Download \texttt{plate.png} from the website and import it as the degraded image $\mZ$. It's a severely degraded image of a license plate. Experiment with $p$ and $\lambda$ and do your best to identify all of the characters on the plate. You might want to use different parameter values for different characters.

For this objective, submit the following:
\begin{itemize}
\item One or more PNG images, showing your best reconstructions for the characters on the plate. The title of each image should show the values of $p$ and $\lambda$ used to generate it.
\end{itemize}

\end{description}

\end{document}