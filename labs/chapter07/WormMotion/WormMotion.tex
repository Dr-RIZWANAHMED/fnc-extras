\documentclass[11pt,twoside]{article}

\usepackage[headings]{fullpage}
\usepackage[utopia]{mathdesign}

\pagestyle{myheadings}
\markboth{Eigenworms}{Eigenworms}

\usepackage{amsmath}
\usepackage{bm}

\newcommand{\nat}{\mathbb{N}}          % Natural numbers
\newcommand{\integer}{\mathbb{Z}}      % Integers
\newcommand{\real}{ {\mathbb{R}} }     % Reals
\newcommand{\float}{ {\mathbb{F}} }     % Reals
\newcommand{\rmn}[2]{ \mathbb{R}^{#1\times#2} }     % Reals
\newcommand{\complex}{ {\mathbb{C}} }  % Complex
\newcommand{\macheps}{\ensuremath \varepsilon_{\text{mach}}}

\renewcommand{\Re}{\operatorname{Re}}
\renewcommand{\Im}{\operatorname{Im}}

% Boldface vectors
\newcommand{\bff}{\bm{f}}
\newcommand{\bfF}{\bm{F}}
\newcommand{\bfw}{\bm{w}}
\newcommand{\bfv}{\bm{v}}
\newcommand{\bfe}{\bm{e}}
\newcommand{\bfc}{\bm{c}}
\newcommand{\bfp}{\bm{p}}
\newcommand{\bfq}{\bm{q}}
\newcommand{\bfr}{\bm{r}}
\newcommand{\bfs}{\bm{s}}
\newcommand{\bfu}{\bm{u}}
\newcommand{\bfb}{\bm{b}}
\newcommand{\bfx}{\bm{x}}
\newcommand{\bfy}{\bm{y}}
\newcommand{\bfg}{\bm{g}}
\newcommand{\bfh}{\bm{h}}
\newcommand{\bfz}{\bm{z}}
\newcommand{\bfa}{\bm{a}}
\newcommand{\bft}{\bm{t}}
\newcommand{\bfd}{\bm{d}}
\newcommand{\bfalpha}{\bm{\alpha}}
\newcommand{\bfeps}{\bm{\varepsilon}}
\newcommand{\bfdelta}{\bm{\delta}}
\newcommand{\bfzero}{\bm{0}}
\newcommand{\eye}[1]{\bfe_{#1}}

% Boldface matrix
\newcommand{\m}[1]{\bm{#1}}
\newcommand{\mA}{\m{A}}
\newcommand{\mL}{\m{L}}
\newcommand{\mF}{\m{F}}
\newcommand{\mU}{\m{U}}
\newcommand{\mJ}{\m{J}}
\newcommand{\mP}{\m{P}}
\newcommand{\mQ}{\m{Q}}
\newcommand{\mR}{\m{R}}
\newcommand{\mD}{\m{D}}
\newcommand{\mS}{\m{S}}
\newcommand{\mB}{\m{B}}
\newcommand{\mC}{\m{C}}
\newcommand{\mE}{\m{E}}
\newcommand{\mG}{\m{G}}
\newcommand{\mH}{\m{H}}
\newcommand{\mV}{\m{V}}
\newcommand{\mW}{\m{W}}
\newcommand{\mX}{\m{X}}
\newcommand{\mZ}{\m{Z}}
\newcommand{\mK}{\m{K}}
\newcommand{\mM}{\m{M}}

\newcommand{\meye}{\m{I}}

\newcommand{\ee}[1]{\times 10^{#1}}
\newcommand{\jac}[2]{\frac{\bfd \bm{#1}}{\bfd \bm{#2}}}
\newcommand{\diag}{\operatorname{diag}}
\newcommand{\fl}{\operatorname{fl}}
\newcommand{\circop}[1]{\makebox[0pt][l]{$\bigcirc$}\hspace{1pt}#1}
\newcommand{\myvec}{\operatorname{vec}}
\newcommand{\unvec}{\operatorname{unvec}}
\newcommand{\kron}[2]{#1 \otimes #2}


\begin{document}

\begin{center}
  \bf Early bird gets the eigenworm
\end{center}

The idea and data for this lab come from Stephens \textit{et al.}, ``Dimensionality and Dynamics in the Behavior of \textit{C.\ elegans},'' \textit{PLoS Comput Biol}, 2008. In this work the authors captured video of worms moving as they were subjected to stimuli (from ``standard'' to ``painful''). Using image processing, they found 100 points representing a path along the back of single worm within each frame of the video and computed a representation independent of rotation and translation using the tangents to the path. The result is a $100\times n$ matrix of tangent angles for $n$ frames of video. They used $n=56200$.

The authors then noted that dimension reduction by the SVD is extraordinarily successful for this data set; they showed that the data are well characterized by a small number of ``eigenworms''.\footnote{One often sees the ``eigen-'' prefix used in this context, but the SVD is a more fundamental description of the mathematics.} This makes some sense, as the motions are constrained a great deal by anatomy and kinematics. 

Suppose $\m{T}$ is the matrix of angles; i.e., column $\bft_j$ is the vector of angles in the $j$th video frame. Suppose we have the full SVD $\m{T}=\m{U} \mS \mV^T$. This would make $\mV$ an $n\times n$ matrix, which would not fit in memory, so we have to use a thin SVD. In the text we only did this for an $m\times n$ matrix with $m>n$, which is not the case for $\m{T}$. However, $\m{T}^T=\m{V} \mS^T \mU$ does have a thin form in which $\m{T}^T=\widehat{\mV} \widehat{\mS}^T \mU$, where $\widehat{\mV}$ is only $n\times 100$ and $\widehat{\mS}$ is $100\times 100$. Finally, this gives us
\begin{equation}
  \label{eq:reducesvd}
  \m{T}=\mU \widehat{\mS} \widehat{\mV}^T,
\end{equation}
the thin SVD we need. 

Observe that
\begin{equation}
  \m{T} = \sum_{k=1}^{100} \sigma_k \bfu_k \bfv_k^T.
\end{equation}
Because the singular values are always in decreasing order, we may approximate the original matrix by
\begin{equation}
  \m{T}_r = \sum_{k=1}^{r} \sigma_k \bfu_k \bfv_k^T,
\end{equation}
for some rank $r\ll 100$.  The range of this matrix is spanned by $\bfu_1,\ldots,\bfu_r$, which are the eigenworms. A good way to express the proportion of $\m{T}$ that is captured by $\m{T}_r$ is the ratio
\begin{equation}
  \tau_r = \frac{\|\mathbf{s}_r\|^2_2}{\|\mathbf{s}_{100}\|^2_2},
\end{equation}
where $\mathbf{s}_k$ is the vector $\bigl[ \sigma_1,\ldots,\sigma_k \bigr]$.

One use of the eigenworms is to create a compact representation of the data. A column $\bft_j$ of the original data can be expressed in terms of its closest approximation as a linear combination of the eigenworms:
\[
  \bft_j \approx c_1 \bfu_1  + \cdots + c_r \bfu_r = \mU_r \bfc,
\]
where $\mU_r$ is $100\times r$. This is a least squares problem, but by orthogonality its solution is $\bfc=\mU_r^T\bft_j$. The $r$ values in this vector give the components of $\bft_j$ in the principal eigenworm directions and could be used as a low-dimensional representation for further analysis.  

\subsection*{Goals}

Given the data matrix, you will perform the SVD analysis, find a reasonable value for the cutoff rank $r$ using the coefficients $\tau_k$, and extract the eigenworms. 

\subsection*{Preparation}

Read section 7.5. Read the online help for \texttt{svd}; it uses the term ``economy size'' for the thin SVD. 

\subsection*{Procedure}

\begin{enumerate}
\item Load the \texttt{shapes.mat} file from the assignment site. It has the matrix \texttt{T}. 

\item On one graph, plot the first three columns of $\m{T}$. (These are tangent angles of the worm's body as a function of arc length.) 

\item Using \texttt{svd}, compute the three matrices in the thin SVD~\eqref{eq:reducesvd}.

\item Let \texttt{s} be the vector of singular values. Using it, plot $1-\tau_r$ versus $r$ on a semi-log scale, for $r=1,\dots,100$. From this plot it should be clear that $r=4$ is a compelling choice. Compute and print out the value of $\tau_4$.
  
\item In a 2-by-2 subplot grid, plot the first 4 eigenworms.

\item For the first three columns of $\mT$, plot the best approximation of each column by the leading 4 eigenworms. (The results will be much smoother curves than in step~1.) 
\end{enumerate}

\subsection*{Extra}
The data has been presented as $\theta(s)$, i.e.\ tangent angle as a function of arclength. The tangent and normal vectors to the curve are given by
\begin{equation}
  \begin{split}
    \bft'(s) &= \frac{d\theta}{ds} \mathbf{n}(s),\\
    \mathbf{n}'(s) &= -\frac{d\theta}{ds} \mathbf{t}(s), \\
    \bft(0) &= \bigl[1,0\bigr], \quad \mathbf{n}(0) = \bigl[ 0,1 \bigr].
  \end{split}
\end{equation}
The position vector $\bfr(s)=\bigl[x(s),y(s)\bigr]$ is then given by $\bfr'(s)=\bft(s)$, $\bfr(0)=\bigl[ 0,0 \bigr]$. For each of the four eigenworms plotted in step~5, solve these equations and plot $\bfr(s)$. (Hint: You may either discretize the ODEs using Euler's method and then plug in the computed $\theta$, or you may interpolate the computed $\theta$ and plug that into an IVP solver. High accuracy is not a concern here.)


\end{document}

%%% Local Variables: 
%%% mode: latex
%%% TeX-master: t
%%% End: 
