\documentclass[11pt,twoside]{article}

\usepackage[headings]{fullpage}
\usepackage[utopia]{mathdesign}

\pagestyle{myheadings}
\markboth{Spring fever}{Spring fever}

\usepackage{amsmath}
\usepackage{bm}

\newcommand{\nat}{\mathbb{N}}          % Natural numbers
\newcommand{\integer}{\mathbb{Z}}      % Integers
\newcommand{\real}{ {\mathbb{R}} }     % Reals
\newcommand{\float}{ {\mathbb{F}} }     % Reals
\newcommand{\rmn}[2]{ \mathbb{R}^{#1\times#2} }     % Reals
\newcommand{\complex}{ {\mathbb{C}} }  % Complex
\newcommand{\macheps}{\ensuremath \varepsilon_{\text{mach}}}

\renewcommand{\Re}{\operatorname{Re}}
\renewcommand{\Im}{\operatorname{Im}}

% Boldface vectors
\newcommand{\bff}{\bm{f}}
\newcommand{\bfF}{\bm{F}}
\newcommand{\bfw}{\bm{w}}
\newcommand{\bfv}{\bm{v}}
\newcommand{\bfe}{\bm{e}}
\newcommand{\bfc}{\bm{c}}
\newcommand{\bfp}{\bm{p}}
\newcommand{\bfq}{\bm{q}}
\newcommand{\bfr}{\bm{r}}
\newcommand{\bfs}{\bm{s}}
\newcommand{\bfu}{\bm{u}}
\newcommand{\bfb}{\bm{b}}
\newcommand{\bfx}{\bm{x}}
\newcommand{\bfy}{\bm{y}}
\newcommand{\bfg}{\bm{g}}
\newcommand{\bfh}{\bm{h}}
\newcommand{\bfz}{\bm{z}}
\newcommand{\bfa}{\bm{a}}
\newcommand{\bft}{\bm{t}}
\newcommand{\bfd}{\bm{d}}
\newcommand{\bfalpha}{\bm{\alpha}}
\newcommand{\bfeps}{\bm{\varepsilon}}
\newcommand{\bfdelta}{\bm{\delta}}
\newcommand{\bfzero}{\bm{0}}
\newcommand{\eye}[1]{\bfe_{#1}}

% Boldface matrix
\newcommand{\m}[1]{\bm{#1}}
\newcommand{\mA}{\m{A}}
\newcommand{\mL}{\m{L}}
\newcommand{\mF}{\m{F}}
\newcommand{\mU}{\m{U}}
\newcommand{\mJ}{\m{J}}
\newcommand{\mP}{\m{P}}
\newcommand{\mQ}{\m{Q}}
\newcommand{\mR}{\m{R}}
\newcommand{\mD}{\m{D}}
\newcommand{\mS}{\m{S}}
\newcommand{\mB}{\m{B}}
\newcommand{\mC}{\m{C}}
\newcommand{\mE}{\m{E}}
\newcommand{\mG}{\m{G}}
\newcommand{\mH}{\m{H}}
\newcommand{\mV}{\m{V}}
\newcommand{\mW}{\m{W}}
\newcommand{\mX}{\m{X}}
\newcommand{\mZ}{\m{Z}}
\newcommand{\mK}{\m{K}}
\newcommand{\mM}{\m{M}}

\newcommand{\meye}{\m{I}}

\newcommand{\ee}[1]{\times 10^{#1}}
\newcommand{\jac}[2]{\frac{\bfd \bm{#1}}{\bfd \bm{#2}}}
\newcommand{\diag}{\operatorname{diag}}
\newcommand{\fl}{\operatorname{fl}}
\newcommand{\circop}[1]{\makebox[0pt][l]{$\bigcirc$}\hspace{1pt}#1}
\newcommand{\myvec}{\operatorname{vec}}
\newcommand{\unvec}{\operatorname{unvec}}
\newcommand{\kron}[2]{#1 \otimes #2}


\begin{document}

\begin{center}
  \bf Spring fever
\end{center}

 The second-order ODE
\[
  y'' + \gamma y' + 2y = \sin(t)
\]
describes the simple harmonic oscillation of (for example) a mass
suspended by a spring and subjected to a periodic force. The constant
$\gamma$ models damping due to friction. Neglecting the friction, the
natural frequency of oscillation is $\omega_0=\sqrt{2}$, and the total
motion combines the natural oscillation with that of the driving
external force. If $\gamma>0$, however, the unforced solution dies
off, and the oscillation is driven entirely by the external force.

One complication in assessing variable-step automatic integrators, such as the RK23 method described in the text, is that we mathematically describe convergence as being at the rate $O(h^p)$ for a fixed step size $h$. If we look instead at the total number $n$ of steps taken by the method over time interval $[a,b]$, then the ``average step size'' is $(b-a)/n$ and we can hope for convergence of $O(n^{-p})$. 

\subsection*{Goals}

You will use the driven spring problem to assess the convergence of the \texttt{rk23} function given in the text.

\subsection*{Preparation}

Read section 6.5. Write the ODE as the first-order system $\bfu' = \bff(t,\bfu)$. (That is, write out exactly what the function $\bff$ is.) 

\subsection*{Procedure}

Download the script template and complete it to perform the following steps. Throughout this lab, use $0\le t \le 20$ and $y(0)=0$, $y'(0)=0$.

\begin{enumerate}
\item Consider the ODE as a first-order system, $\bfu'=\bff(t,\bfu)$. In a separate file, write a function 
\begin{verbatim}
function f = myspring(t,u,gamma)
\end{verbatim} that defines the ODE by returning the value of $\bff$ for any given $t$ and $\bfu$. 

\item Define \texttt{u0} as the initial condition of the first-order system and let $\gamma=10$. Create a reference solution using the syntax
\begin{verbatim}
dudt = @(t,u) myspring(t,u,10);
opt = odeset('reltol',1e-13,'abstol',1e-13);
[t,u_ref] = ode15s(dudt,[0 20],u0,opt);
\end{verbatim}
  where \texttt{u0} is defined appropriately. Plot the solution $y(t)$ as a function of $t$, using a title and labeled axes. In the title or as text on the plot, include the number of time steps that were taken by the solver.

\item Use the adaptive \texttt{rk23} solver from the text to solve the
  problem with the \texttt{tol} argument set to $10^{-4}$. Make a
  phase plot of the solution (i.e., with $y$ on the horizontal axis
  and $y'$ on the vertical axis). Don't forget labels and a title.

\item Use \texttt{rk23} to solve the problem with each of the error
  tolerances $10^{-2}$, $10^{-3}$, \ldots, $10^{-12}$ in turn. After each
  solution, record in two vectors the number of time steps
  taken, $n$, and the global error, $E_n=\| \bfu_{\text{ref}}(20) - \bfu(20)\|$. (\emph{Important}: This is at the final time $t=40$, \emph{not} the solution in the 30th row of the output from \texttt{rk23}.) Output a table with each row giving the tolerance, $n$, and $E_n$. 

\item On a new graph, make a log-log plot of $E_n$ versus $n$. The points
    should mostly lie close to a straight line. Add to the plot two
    straight lines indicating perfect second- and third-order convergence,
    $E_n=n^{-2}$ and $E_n=n^{-3}$.
    
\item Now let $\gamma=5000$. Repeat steps 2--3.
    
\item Repeat steps 4--5 (still with $\gamma=5000$). The results will be very different.
\end{enumerate}

\subsection*{Discussion}

The time stepping in \texttt{rk23} is supposed to be third-order
accurate. But that is a statement about the mathematical limit
$h\to 0$, or $n\to \infty$. For particular finite values of $n$, we
might observe something close to the ideal $E_n=Cn^{-3}$, but just
about anything is possible. 

In fact, for this spring problem with $\gamma=5000$, the time step
size adaptation is being constrained by something other than
third-order accuracy. The mathematics behind this observation is
considered in Chapter 10.




\end{document}

%%% Local Variables: 
%%% mode: latex
%%% TeX-master: t
%%% End: 
