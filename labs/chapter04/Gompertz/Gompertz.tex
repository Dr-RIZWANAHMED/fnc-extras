\documentclass[11pt]{article}

\usepackage{amsmath,amsthm}
\usepackage[headings]{fullpage}
\usepackage[utopia]{mathdesign}
\usepackage{color}

\pagestyle{myheadings}
\markboth{MATH426/CISC410}{MATH426/CISC410}

\usepackage{amsmath}
\usepackage{bm}

\newcommand{\nat}{\mathbb{N}}          % Natural numbers
\newcommand{\integer}{\mathbb{Z}}      % Integers
\newcommand{\real}{ {\mathbb{R}} }     % Reals
\newcommand{\float}{ {\mathbb{F}} }     % Reals
\newcommand{\rmn}[2]{ \mathbb{R}^{#1\times#2} }     % Reals
\newcommand{\complex}{ {\mathbb{C}} }  % Complex
\newcommand{\macheps}{\ensuremath \varepsilon_{\text{mach}}}

\renewcommand{\Re}{\operatorname{Re}}
\renewcommand{\Im}{\operatorname{Im}}

% Boldface vectors
\newcommand{\bff}{\bm{f}}
\newcommand{\bfF}{\bm{F}}
\newcommand{\bfw}{\bm{w}}
\newcommand{\bfv}{\bm{v}}
\newcommand{\bfe}{\bm{e}}
\newcommand{\bfc}{\bm{c}}
\newcommand{\bfp}{\bm{p}}
\newcommand{\bfq}{\bm{q}}
\newcommand{\bfr}{\bm{r}}
\newcommand{\bfs}{\bm{s}}
\newcommand{\bfu}{\bm{u}}
\newcommand{\bfb}{\bm{b}}
\newcommand{\bfx}{\bm{x}}
\newcommand{\bfy}{\bm{y}}
\newcommand{\bfg}{\bm{g}}
\newcommand{\bfh}{\bm{h}}
\newcommand{\bfz}{\bm{z}}
\newcommand{\bfa}{\bm{a}}
\newcommand{\bft}{\bm{t}}
\newcommand{\bfd}{\bm{d}}
\newcommand{\bfalpha}{\bm{\alpha}}
\newcommand{\bfeps}{\bm{\varepsilon}}
\newcommand{\bfdelta}{\bm{\delta}}
\newcommand{\bfzero}{\bm{0}}
\newcommand{\eye}[1]{\bfe_{#1}}

% Boldface matrix
\newcommand{\m}[1]{\bm{#1}}
\newcommand{\mA}{\m{A}}
\newcommand{\mL}{\m{L}}
\newcommand{\mF}{\m{F}}
\newcommand{\mU}{\m{U}}
\newcommand{\mJ}{\m{J}}
\newcommand{\mP}{\m{P}}
\newcommand{\mQ}{\m{Q}}
\newcommand{\mR}{\m{R}}
\newcommand{\mD}{\m{D}}
\newcommand{\mS}{\m{S}}
\newcommand{\mB}{\m{B}}
\newcommand{\mC}{\m{C}}
\newcommand{\mE}{\m{E}}
\newcommand{\mG}{\m{G}}
\newcommand{\mH}{\m{H}}
\newcommand{\mV}{\m{V}}
\newcommand{\mW}{\m{W}}
\newcommand{\mX}{\m{X}}
\newcommand{\mZ}{\m{Z}}
\newcommand{\mK}{\m{K}}
\newcommand{\mM}{\m{M}}

\newcommand{\meye}{\m{I}}

\newcommand{\ee}[1]{\times 10^{#1}}
\newcommand{\jac}[2]{\frac{\bfd \bm{#1}}{\bfd \bm{#2}}}
\newcommand{\diag}{\operatorname{diag}}
\newcommand{\fl}{\operatorname{fl}}
\newcommand{\circop}[1]{\makebox[0pt][l]{$\bigcirc$}\hspace{1pt}#1}
\newcommand{\myvec}{\operatorname{vec}}
\newcommand{\unvec}{\operatorname{unvec}}
\newcommand{\kron}[2]{#1 \otimes #2}

\begin{document}

\begin{center}
  \bf A fitting challenge
\end{center}

Some phenomena, like the growth of tumors, can be predicted by theory to follow a \textbf{Gompertz function},
\begin{equation}
  \label{g}
  g(t) = A e^{-b e^{-ct}},
\end{equation}
where $A,b,c$ are positive parameters. As $t$ varies over the real line, $g$ increases from 0 to $A$. 

Suppose you have data $(t_i,z_i)$, $i=1,\dots,m$, that you believe follow a Gompertz curve with unknown parameters. Taking the log, we get
\begin{equation}
  \label{logdata}
  \log(z_i) \approx (\log A) - b e^{-c t_i}.
\end{equation}
We let $y_i=\log z_i$ and let $a=\log A$. Define a function $\bff$ whose components are
\begin{equation}
	\label{f}
	f_i(a,b,c) = a - b e^{-c t_i} - y_i.
\end{equation}
In order to fit the data, we seek to minimize $\|\bff\|_2$ as a function of $a$, $b$, and $c$, which together we call the vector $\bfx$. This is a nonlinear least squares problem that can be solved by the Levenberg iteration.

\subsection*{Preparation}

Read section 4.7. 


\subsection*{Goals}

You will find a nonlinear least squares fit to given data and use the result to predict the asymptotic value $g(\infty)=A$. 

\subsection*{Procedure}

Download the script template and the file \texttt{gompertz\_data}.

\begin{enumerate}
    \item Load the data file, which has two vectors \texttt{t} and \texttt{z}. Make a plot of $z$ versus $t$. 
     
    \item In a separate file, write a function 
\begin{verbatim}
function f = gomp(x,t,z)
\end{verbatim}
    that returns the $m$-vector defined by~\eqref{f}, given a value of $\bfx=[a;b;c]$ and the vectors \texttt{t} and \texttt{z}.
    
    \item In the script file, use \texttt{levenberg} to find best fitting values for $a,b,c$. 
    
    \item Using the result of the fit, calculate the value of $\lim_{t\to\infty} g(t)$. 
    
    \item On top of the data points, make a plot of the fitting function~\eqref{g} over $0\le t \le 40$. (Remember that the function fits the log of the values, not the values themselves.)
\end{enumerate}


\end{document}

